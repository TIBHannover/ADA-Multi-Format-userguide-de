\documentclass{article}

\usepackage{hyperref}
\usepackage{caption}
\usepackage{graphicx}
                
\usepackage{calc}
                
\newlength{\imgwidth}
                
\newcommand\scaledgraphics[2]{%
                
\settowidth{\imgwidth}{\includegraphics{#1}}%
                
\setlength{\imgwidth}{\minof{\imgwidth}{#2\textwidth}}%
                
\includegraphics[width=\imgwidth,height=\textheight,keepaspectratio]{#1}%
                
}
            
\begin{document}

\title{Schritt 1: Veröffentlichung Git Repo und Website}

\maketitle


Das Git-Repository (Repo) ist der Speicherort Ihrer erstellten Publikation, welcher sich im Internet befindet. Aus dem Repository kann auch eine Präsentations-Website erstellt werden, auf der ausgewählte Inhalte veröffentlicht werden. Wenn Ihr Repository aktualisiert wird, aktualisiert sich auch Ihre Website.

\begin{figure}
\scaledgraphics{d80dc5a5-d385-4a3b-a03a-a752ff2686c9.png}{1}
\caption*{Foto 1: Beispiel für ein Git-Repository (repo)}\label{F86948651}
\end{figure}

\begin{figure}
\scaledgraphics{0aea86b5-871e-4b80-95ca-5b030854029d.png}{1}
\caption*{Foto 2: Beispiel für eine mit GitHub/Lab Pages erstellte Publikationswebsite, die eine Präsentation des obigen Repositorys ist. Der Pfeil zeigt die Links zu den anderen verfügbaren Formaten}\label{F80434841}
\end{figure}


\subsection{Über Git}\label{H9482233}



Das Repository verwendet die \href{https://git-scm.com/}{Git-Technologie}\footnote{Git ist eine Open-Source-Software, auf der sowohl \href{https://git-scm.com/}{GitHub} als auch \href{https://about.gitlab.com/}{GitLab} aufgebaut sind.}, die die Formatierung von Dateien ermöglicht und zum Speichern Ihrer Veröffentlichung verwendet wird.


Das System bietet die Möglichkeit, GitLab CE, GitLab.com oder GitHub zu verwenden. GitLab kann als GitLab.com oder als selbst gehostete Instanz für öffentliche und private Veröffentlichungen oder für die Bereitstellung von Veröffentlichungen für die spätere Übertragung auf ein drittes System, entweder GitHub oder GitLab.com, verwendet werden. Wir verwenden GitLab Community Edition (GitLab CE), eine Open-Source-Software, für die eigenständige Betreuung. GitHub ist keine Open-Source-Software, eignet sich aber für die Verbreitung und Sichtbarkeit von Publikationen.

\begin{figure}
\scaledgraphics{7de3d3b1-1706-4dde-a73d-2bd35c02487f.png}{1}
\caption*{Foto 3: Git logos - Git; GitLab, and; GitHub}\label{F72941601}
\end{figure}


\subsection{Schritt-für-Schritt-Anleitung}\label{H7833693}



Diese Anleitung bezieht sich auf die Verwendung von GitHub. Die Grundsätze sind die gleichen für GitLab.


Mit diesen Schritten können Sie eine Ablage für Ihre Publikation mit der Option zur Erstellung einer Website über Git (Hub/Lab) Pages erstellen.


\subsubsection{Ein Repository erstellen}\label{H7778151}



Um Ihr Repository zu erstellen, verwenden wir ein Vorlagen-Repository.


Sie erstellen ein Repository, das vom Template Repo vorausgefüllt wird, damit Sie später den Inhalt Ihrer Publikation hinzufügen können. Die Repository-Ablage enthält Komponenten zur Erstellung der Website und zur Bereitstellung von Links zu anderen Publikationsformaten, die auf der Website als Links angezeigt werden.


Es gibt eine Reihe von Vorlagen, die Sie verwenden können. Sprechen Sie dies mit Ihrem Publikationsmanager ab, welcher Ihnen sagen kann, welche Vorlage Sie verwenden sollen. Als Beispiel finden Sie hier eine Vorlage auf GitHub aus der \href{https://github.com/TIBHannover/ADA-Book-Template}{ADA-Pipeline}, die von der Deutschen Zentralbibliothek für Technikwissenschaften (TIB) gepflegt wird.


Navigieren Sie zu dem Link für die Vorlage und klicken Sie auf die grüne Schaltfläche "Diese Vorlage verwenden".

\begin{figure}
\scaledgraphics{0ce6c9ea-6817-4c0d-b56b-93b0dcd88ac8.png}{1}
\caption*{Foto 4: Beispiel für ein Vorlagen-Repository. Verwenden Sie das Vorlagen-Repositorium, um Ihr Publikations-Repositorium vorzufüllen}\label{F95809581}
\end{figure}


Wählen Sie dann den Ort, an dem Sie die neue Ablage anlegen möchten, und ihren Namen. Klicken Sie anschließend auf "Repository aus dieser Vorlage erstellen".

\begin{figure}
\scaledgraphics{e61a310c-a521-4f60-81ba-bdfacba7ef99.png}{1}
\caption*{Foto 5: Legen Sie Eigentümer (Standort), Name und Beschreibung fest. Dann speichern}\label{F59276791}
\end{figure}


\textbf{Wo speichern Sie Ihre Ablage?} In GitHub können Sie Repos mit Organisationen oder in Ihrem persönlichen Konto speichern, wählen Sie dies unter dem Feld "Eigentümer" aus.


\textbf{Wie benennen Sie Ihre Ablage?} Der Name des Repos ist sein Anzeigename und seine URL-Adresse. Es empfiehlt sich, einen Namen zu wählen, der mit anderen Veröffentlichungen übereinstimmt, z. B. ein Kurztitel oder sogar ein Akronym. Beachten Sie, dass im Namen nur Kleinbuchstaben verwendet werden sollten, da bei der URL Groß- und Kleinschreibung beachtet wird. Die Namen können jederzeit geändert werden, allerdings werden dadurch auch die zugehörigen URLs geändert.


\textbf{Hinweis:} Die Namen der Repos können jederzeit geändert werden, allerdings wird dadurch die URL der Website auf den neuen Namen geändert. Es gilt daran zu denken, die URL an anderen Stellen zu aktualisieren, an denen Sie die URL-Adresse bereits verwendet haben.


\textbf{Andere Einstellungen:} Sie können dem Projektarchiv eine Beschreibung geben; die Voreinstellung ist, dass das Projektarchiv öffentlich ist. Klicken Sie dann auf die grüne Schaltfläche, um Ihre Einstellungen zu speichern.


Herzlichen Glückwunsch, Sie haben nun Ihre Ablage erstellt und einen Ort gefunden, an dem Sie Ihre Publikationen speichern können.


\subsubsection{Erstellen Sie eine GitHub/Lab Pages-Website}\label{H3055126}



GitHub bietet einen Dienst namens „GitHub Pages“ an. Damit werden kostenlose Websites unter der Domain github.io erstellt, oder Sie können Ihren eigenen benutzerdefinierten Bereich verwenden. Das Standard-URL-Adressmuster ist https://organisation-name.github.io/publication-name/. Der Inhalt Ihres Projektarchivs wird auf der Website unter der angegebenen URL verfügbar sein.


Hinweis: Websites können mit benutzerdefinierten Domainnamen versehen werden. Hierfür müssen Sie die GitHub-Dokumentation konsultieren, um die Funktion zu aktivieren.


Dies ist ein zweiteiliger Prozess.


\subsubsection{Teil 1: Aktivieren Sie die Erstellung von GitHub/Lab Pages-Websites}\label{H327158}


\begin{figure}
\scaledgraphics{e61a310c-a521-4f60-81ba-bdfacba7ef99.png}{1}
\caption*{Foto 6: Schalten Sie Seiten ein. Besuchen Sie die Registerkarte "Einstellungen"; den linken Menüpunkt "Seiten"; setzen Sie ihn auf "Haupt" und "Stamm".}\label{F96677601}
\end{figure}


Navigieren Sie zu "Einstellungen" in den oberen horizontalen Registerkartenoptionen. Wählen Sie in den Einstellungen im linken Menü "Seiten" aus. Nehmen Sie im Dialog auf der Hauptseite die folgenden Einstellungen vor: Wählen Sie den Zweig "Main“; danach wählen Sie den Ordner "Root“, und klicken Sie auf "Speichern“. Damit ist die Erstellung der Seite abgeschlossen und Sie erhalten eine URL für Ihre Website. Kopieren Sie die URL und verwenden Sie sie, um die Adresse auf die Vorderseite des Repo einzufügen.

\begin{figure}
\scaledgraphics{c95e502b-6265-4634-98c4-a310aaa0ba5b.png}{1}
\caption*{Foto 7: Schalten Sie Seiten ein. Besuchen Sie die Registerkarte "Einstellungen"; den linken Menüpunkt "Seiten"; setzen Sie ihn auf "Haupt" und "Stamm".}\label{F72487461}
\end{figure}


Um den Namen der Website auf der Vorderseite Ihres Repo einzufügen, gehen Sie zunächst zur Vorderseite des Repo, indem Sie auf der linken Seite der horizontalen Registerkarten Ihres Repo auf "Code“ klicken. Klicken Sie im Anschluss auf der rechten Seite auf das Zahnrad neben "About“. Hier können Sie die URL einfügen und speichern.

\begin{figure}
\scaledgraphics{217faf4f-364b-4d7f-b5cd-ff3483ed5e08.png}{1}
\caption*{Foto 8: Sie können den Namen der Publikation und die URL-Adresse in die Infobox eingeben, die dann auf der Vorderseite des Repos angezeigt wird.}\label{F41996561}
\end{figure}


Sie haben nun eine Website und die Adresse erscheint oben rechts.

\begin{figure}
\scaledgraphics{f4544e29-1d39-4948-a754-7554fcddb8ad.png}{1}
\caption*{Foto 9: Nach der Eingabe werden der Name und die URL-Adresse oben rechts angezeigt.}\label{F98414411}
\end{figure}


Ihre Website wird wie folgt aussehen. Derzeit enthält die Website Benchmark-Inhalte, um anzuzeigen, dass die Layout-Funktionen korrekt funktionieren. Dieser Inhalt wird mit Ihren Inhalten ersetzt, sobald Sie Ihre Publikation veröffentlicht haben.

\begin{figure}
\scaledgraphics{29112ec9-518c-4a92-a6fc-237fc26b41b8.png}{1}
\caption*{Foto 10: Zu Beginn wird Ihre Website mit Benchmark-Testinhalten aus der Vorlage versehen sein. Später, bei der Ausgabe aus Fidus, wird dies überschrieben.}\label{F91564721}
\end{figure}


\subsubsection{Teil 2: Anzeigen von Multiformat-Inhalten auf GitHub/Lab-Seiten}\label{H5915252}



Um die paginierte Webversion Ihrer Publikation zu aktivieren, muss die Repo-Adresse zur setup.json-Datei auf der obersten Ebene Ihres Repos hinzugefügt werden.

\begin{figure}
\scaledgraphics{937e8cc0-b575-4ebe-acc3-4faf51c83ced.png}{1}
\caption*{Foto 11: Suchen Sie die Datei setup.json in der obersten Ebene Ihres Repos. Klicken Sie darauf, um sie anzuzeigen und zu bearbeiten}\label{F8474601}
\end{figure}


Bearbeiten Sie die Datei „setup.json“ und fügen Sie den Organisationsnamen und den Repo-Namen in Zeile 3 ein, speichern Sie anschließend unten auf der Seite, "repoURL": "https://github.com/organisation-name/publication-name/".

\begin{figure}
\scaledgraphics{8e9e4c18-b941-4173-b655-fe716a42a060.png}{1}
\caption*{Foto 12: Zum Bearbeiten klicken Sie auf das Bleistiftsymbol oben rechts. Bearbeiten Sie dann Zeile 3 und ändern Sie die Adresse Ihres Repos}\label{F65883321}
\end{figure}


Alle Schritte Ihrer Git-Einrichtung sind nun abgeschlossen.


\subsection{Schritt 1 ist abgeschlossen: Wie geht es weiter?}\label{H2358897}



Jetzt, wo Sie Ihr Repo und Ihre Website eingerichtet haben, richten Sie als Nächstes ein Buchprojekt in Fidus Writer ein und verbinden es mit Ihrem Git-Repo, damit Sie Buchdateien aus Fidus Writer in Git übertragen können.

\end{document}
