\documentclass{article}

\usepackage{hyperref}
\usepackage{caption}
\usepackage{graphicx}
                
\usepackage{calc}
                
\newlength{\imgwidth}
                
\newcommand\scaledgraphics[2]{%
                
\settowidth{\imgwidth}{\includegraphics{#1}}%
                
\setlength{\imgwidth}{\minof{\imgwidth}{#2\textwidth}}%
                
\includegraphics[width=\imgwidth,height=\textheight,keepaspectratio]{#1}%
                
}
            
\begin{document}

\title{Was Sie für den Einstieg benötigen}

\maketitle


Im Folgenden finden Sie Anweisungen zu den Schritten, die Sie ausführen müssen, bevor Sie mit der Arbeit an einem "Publishing Pipeline"-Projekt beginnen.


\textbf{Alle Benutzer benötigen eine E-Mail-Adresse, welche E-Mails mit URL-Links empfangen kann und bei der Kontoverifizierung im Internet verwendet werden kann.}


Der folgende Inhalt ist nach Benutzertypen geordnet:

\begin{itemize}
\item Mitwirkender an der Veröffentlichung


\item Publikationsmanager


\end{itemize}

\textbf{Erstellung eines Kontos:} Die folgenden Anweisungen beziehen sich auf die Standardmethoden zur Erstellung von Konten. Wenn die aufgelisteten Plattformen "Single Sign On"-Funktionen verwenden, kann die Authentifizierung auch über die Anmeldedaten des Unternehmens oder über Plattformkonten wie GitLab, OAuth-Dienste oder über andere Authentifizierungs- und Autorisierungsinfrastrukturen (AAI) erfolgen.


\subsection{Beitragende zur Veröffentlichung}\label{H9091768}



\subsubsection{Erstellung eines Kontos}\label{H7088114}



\subsubsection{Fidus Writer}\label{H6131222}



Fidus Writer verfügt über drei Verfahren zur Erstellung von Konten, und je nach der von Ihnen verwendeten Instanz haben Sie unterschiedliche Optionen:

\begin{enumerate}
\item Nur einladen


\item Anmelden


\item Einmalige Anmeldung mit Authentifizierung


\end{enumerate}

\textbf{Nur einladen:} Wenn es auf der Website keine Schaltfläche "Anmelden" gibt, erfolgt die Kontoerstellung nur über eine Einladung. Bitte wenden Sie sich an die Website-Manager und beantragen Sie ein Konto. Wenn Sie für ein Konto zugelassen werden, erhalten Sie eine E-Mail mit Anweisungen für den Abschluss des Kontoerstellungsprozesses.


\textbf{Anmelden:} Folgen Sie den Anweisungen zur Kontoerstellung auf der Website. Sobald Sie alle Angaben gemacht haben, erhalten Sie eine E-Mail, um den Anmeldevorgang abzuschließen.


\emph{Wichtiger Hinweis: Sie müssen Ihr Konto verifizieren, indem Sie auf die in Ihrem E-Mail-Posteingang eingegangene E-Mail klicken und den Datenschutzbestimmungen zustimmen, um die Kontoerstellung abzuschließen. Wenn Sie dies nicht tun, kann Ihr Konto nicht erstellt werden.}


\textbf{Single Sign On:} Fidus Writer kann mit Authentifizierungsdaten von anderen Plattformen verwendet werden, welche das OAuth-Protokoll verwenden. Wenn Sie über ein Konto mit einem Authentifizierungsdienst verfügen, der in der von Ihnen verwendeten Fidus Writer-Instanz aufgeführt ist, können Sie sich mit diesem Konto anmelden, unabhängig davon, ob es sich um Ihr Arbeitsplatzkonto oder ein anderes Plattformkonto handelt, zum Beispiel auch ein GitLab-Konto.


Sie können sehen, ob andere Plattform-Logins verwendet werden können, da diese auf der Fidus Writer-Homepage aufgeführt werden. In der unten aufgezeigten Bildschirmaufnahme, sehen Sie eine Beispielseite, die GitHub und GitLab verwendet.

\begin{figure}
\scaledgraphics{3f36357f-61e1-4659-ba2a-735e243dfaf9.png}{1}
\caption*{Verfügbare soziale Konten}\label{F80718861}
\end{figure}


\subsection{Veröffentlichungs-Manager}\label{H558688}



\subsubsection{Erstellung eines Kontos}\label{H3028696}



\subsubsection{Fidus Writer}\label{H98190}



Siehe Anweisungen oben.


\subsubsection{Git}\label{H5911103}



Je nachdem, welche Git-Plattform Sie in der "Veröffentlichungsplattform" verwenden, benötigen Sie ein Konto auf jeder Plattform. Möglicherweise verwenden Sie mehr als eine Git-Plattform. In diesem Fall benötigen Sie ein Konto für jede einzelne Plattform.


Derzeit unterstützte Plattformen sind: GitLab CE, GitLab.com, und GitHub.com (2022).


\subsubsection{GitLab-Konto}\label{H7772100}



Das gleiche Verfahren wird für GitLab Community Edition (GitLab CE) oder für GitLab.com verwendet.


Folgen Sie den Anweisungen auf der Website hier \href{https://gitlab.com/users/sign_up}{https://gitlab.com/users/sign\_up}


\subsubsection{GitHub-Konto}\label{H8045385}



Befolgen Sie die Anweisungen auf der Website hier \href{https://github.com/signup}{https://github.com/signup}

\end{document}
