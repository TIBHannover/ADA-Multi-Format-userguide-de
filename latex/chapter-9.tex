\documentclass{article}

\begin{document}

\title{Glossar}

\maketitle


Begriffe, die in der Publishing-Pipeline verwendet werden.


\textbf{Automatischer Satz} - Die Verwendung von heuristischen Maschinenregeln für den Satz einer Publikation.


\textbf{Creative-Commons-Lizenz} - Eine Creative-Commons-Lizenz (CC) ist eine von mehreren öffentlichen Urheberrechtslizenzen, die die freie Verbreitung eines ansonsten urheberrechtlich geschützten "Werks" ermöglichen.


\textbf{Git cryptographic ID} - Eine Möglichkeit, mit Hilfe der Git Commit ID (SHA) einen eindeutigen Identifikator für Inhalte zu vergeben, die mit Git gespeichert wurden.


\textbf{Digitale Souveränität} - Digitale Souveränität beschreibt den Grad der Kontrolle, den eine Person, eine Organisation oder eine Regierung über die Daten hat, die sie auf lokalen oder Online-Plattformen erzeugt und mit denen sie arbeitet.


\textbf{DOI} - Ein DOI (Digital Object Identifier) ist eine eindeutige und unveränderliche Zeichenfolge, die einer Online-Publikation und ihren Unterkomponenten - Kapitel, Bilder, Videos usw. - zugewiesen wird.


\textbf{FAIR / FAIR Data} - FAIR-Daten sind Daten, welche den Prinzipien der Auffindbarkeit, Zugänglichkeit, Interoperabilität und Wiederverwendbarkeit (FAIR) entsprechen.


\textbf{Fidus Writer} - Fidus Writer ist ein kollaborativer Online-Editor, der speziell für Akademiker entwickelt wurde, welche Zitate und/oder Formeln verwenden müssen.


\textbf{Git} - Git ist ein freies und Open-Source "verteiltes Versionskontrollsystem". Eine verteilte Versionskontrolle ist eine Form der Versionskontrolle (Verwaltung von Änderungen), bei der die gesamte Codebasis, einschließlich ihrer vollständigen Historie, auf dem Computer eines jeden Entwicklers wiedergegeben wird.


\textbf{GitHub} - GitHub ist ein Internet-Hosting-Service für Softwareentwicklung und Versionskontrolle mit Git.


\textbf{GitHub Pages} - GitHub Pages ist ein statischer Webhosting-Dienst.


\textbf{GitLab} - GitLab ist ein Open-Source-Internet-Hosting-Service für die Softwareentwicklung und Versionskontrolle mit Git.


\textbf{GitLab Pages} - GitLab Pages ist ein statischer Webhosting-Dienst für die Veröffentlichung aus einer Ablage in GitLab.


\textbf{Linked Open Data} - Linked Open Data ist eine Reihe von Designprinzipien für die gemeinsame Nutzung offener, maschinenlesbarer, und miteinander verknüpfter Daten im Web.


\textbf{Multiformat-Publikation} - Veröffentlichung in Formaten wie Print, PDF, Web, E-Book. Bei der Veröffentlichung in mehreren Formaten müssen die Einschränkungen der einzelnen Formate berücksichtigt werden, z. B. ob das Format Tabellen, Videos oder Hyperlinks unterstützen kann. Andere Überlegungen beziehen sich auf die Navigation und die Präsentation, z. B. sind Formate wie das Web in der Regel nicht paginiert, was die Verwendung einer gedruckten Seitenzahl ändert. Schließlich gibt es für jedes Format spezifische Überlegungen zu Metadaten und Vertriebskanälen.


\textbf{Offener Access} - Open Access  ist eine Konvention im akademischen Verlagswesen, Veröffentlichungen frei zugänglich zu machen.


\textbf{Offene Daten} - Offene Daten sind Daten, die offen zugänglich, verwertbar, bearbeitbar und von jedermann für jeden Zweck, auch für kommerzielle Zwecke, gemeinsam nutzbar sind. Offene Daten werden unter einer offenen Lizenz lizenziert.


\textbf{Offene Wissenschaft} - Offene Wissenschaft ist die Bewegung, die darauf abzielt, wissenschaftliche Forschung (einschließlich Veröffentlichungen, Daten, physische Proben und Software) und ihre Verbreitung allen Schichten der Gesellschaft zugänglich zu machen, ob Amateur oder Profi.


\textbf{Open-Source-Software} - Bei einer Open-Source-Software (OSS) handelt es sich um eine Computersoftware, die unter einer Lizenz veröffentlicht wird, in welcher der Urheberrechtsinhaber den Nutzern das Recht einräumt, die Software und ihren Quellcode zu nutzen, zu studieren, zu verändern und an jedermann und für jeden Zweck weiterzugeben.


\textbf{ORCID} - ORCID (Open Researcher and Contributor ID) ist ein nicht-proprietärer alphanumerischer Code zur eindeutigen Identifizierung von Autoren und Mitwirkenden an wissenschaftlicher Kommunikation. ORCID ist ein persistenter Identifikator.


\textbf{Paginiertes Web (Paged Web)} - Paginiertes Web ist die Präsentation von Webseiten als Abfolge von Seiten in Form eines Kodex oder Buches.


\textbf{Persistent identifier (PID)} - Ein persistenter Identifikator ist ein dauerhafter Verweis auf ein Dokument, eine Datei, eine Webseite oder ein anderes Objekt.


\textbf{Publikationsfertige Ausgaben} - Eine publikationsfertige Ausgabe (PRO) bedeutet, dass das Format für eine professionelle Veröffentlichung bereit ist, einschließlich Schriftsatz, Metadaten und anderer Formatierungen und Einstellungen. Viele Systeme können Dateien in einem bestimmten Format speichern, z. B. als HTML oder PDF - das bedeutet jedoch nicht, dass es professionell genutzt werden kann. Microsoft Word kann zwar als HTML oder PDF speichern, aber es macht aus den formatierten Dateien keine fertigen Publikationen, die für den Vertrieb geeignet sind.


\textbf{Repository / Repo (Git Repo)} - Repositories in GIT enthalten eine Sammlung von Dateien verschiedener Versionen eines Projekts. Diese Dateien werden aus dem Repository auf den lokalen Server des Benutzers importiert, um weitere Aktualisierungen und Änderungen am Inhalt der Datei vorzunehmen. Ein VCS oder Versionskontrollsystem wird verwendet, um diese Versionen zu erstellen und sie an einem bestimmten Ort, dem Repository, zu speichern.


\textbf{ROR} - ROR ist ein von der Gemeinschaft geleitetes Projekt zur Entwicklung eines offenen, nachhaltigen, nutzbaren und eindeutigen Identifikators für jede Forschungseinrichtung in der Welt. ROR ist ein persistenter Identifikator (PID).


\textbf{Single-Source-Publishing} - Single-Source-Publishing ist eine Methode zur Verwaltung von Inhalten, die es ermöglicht, denselben Quellinhalt in verschiedenen Medien und mehr als einmal zu verwenden.


\textbf{Versionierung (Git)} - Versionskontrolle, die Verwaltung von Änderungen an Dokumenten, Computerprogrammen, großen Websites und anderen Informationssammlungen.


\textbf{Vivliostyle (CSS Typesetting)} - Ein Open-Source-Projekt für ein neues Schriftsatzsystem, welches für das digitale und Web-Publishing geeignet ist. Zudem basiert es auf der neuesten Web-Standard-Technologie.

\end{document}
